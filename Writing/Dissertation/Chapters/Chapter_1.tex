% !TEX root = ../HonoursThesisTemplate.tex

\chapter{Echo State Networks}


    \begin{itemize}
        \item First introduced by Jaeger (2001). %TODO reference
        \item Type of recurrent neural network.
        \item Commonly studied form of reservoir computing.
        \item A form of supervised learning.
        \item Most weights are fixed and only some are fitted.
        \item Is computationally efficient compared to other neural networks.
    \end{itemize}

    








    State update equation:
    \[
        \mathbf{s}(t + 1) = f_{act}(\mathbf{W}_{in}x(t) + \mathbf{W}_{rec}\mathbf{s}(t) + \mathbf{W}_{bias})
    \]
    \begin{itemize}
        % \begin{itemize}
        %     \item Given a sufficiently long input sequence, the current state of the reservoir is determined uniquely by the intervening input and is independent of its initial state.
        % \end{itemize} % TODO reference here
        \item $\mathbf{W}_{in}$, $\mathbf{W}_{rec}$, $\mathbf{W}_{bias}$ and $\mathbf{s}(t)$ are randomly generated according to hyper-parameters.
        % \item $\mathbf{W}_{rec}$ is a randomly generated network weight matrix (e.g. Erdos-Renyi).
        \item Echo state property
        \item $\mathbf{W}_{in}$, $\mathbf{W}_{rec}$ and $\mathbf{W}_{bias}$ are fixed.
        \item Only the readout vector weights is fitted. % TODO reference A practical guide to applying echo state networks
    \end{itemize}

    









    Output equation:
    \[
        y(t) = \mathbf{C}_{out}\mathbf{s}(t)
    \]
    \[
        \mathbf{Y} = \mathbf{C}_{out}\mathbf{S}
    \]

    Output regression with regularisation: % TODO reference [4] in Lucas'
    \[
        \mathbf{C}_{out} = (\mathbf{S}^T \mathbf{S} + \beta \mathbf{I}) \ \mathbf{S}^T \mathbf{Y}
    \]

    \begin{figure}
        
    %     \centering
        \begin{tikzpicture}[scale=0.8]
            \node[fill=black,circle,inner sep=2pt,label=left:$x(t)$] (input) at (0,0) {};
            \node[fill=black,circle,inner sep=2pt,label=right:$y(t)$] (output) at (10,0) {};
            
            
            \coordinate (res_anchor) at (2,3);
            \node[fill=black,circle,inner sep=2pt] (res1) at ({5-0.5},{0.8}) {};
            \node[fill=black,circle,inner sep=2pt] (res2) at ({5},{-1}) {};
            \node[fill=black,circle,inner sep=2pt] (res3) at ({5},{0.15}) {};
            \node[fill=black,circle,inner sep=2pt] (res4) at ({5+1},{-0.1}) {};
            \node[fill=black,circle,inner sep=2pt] (res5) at ({5-0.75},{-0.5}) {};
            \node[fill=black,circle,inner sep=2pt] (res6) at ({5-1.25},{0.4}) {};
            \node[fill=black,circle,inner sep=2pt] (res7) at ({5+0.65},{0.75}) {};
            \node[fill=black,circle,inner sep=2pt] (res8) at ({5+1},{-0.9}) {};
            \node[fill=black,circle,inner sep=2pt] (res9) at ({5-1.25},{-1}) {};
            \node[fill=black,circle,inner sep=2pt] (res10) at ({5-1.75},{-0.25}) {};
        
            \foreach \i in {1,...,10}
                \draw[gray,line width=0.1] (input) -- (res\i);
            
            \foreach \i in {1,...,5}
                \foreach \j in {1,...,5} {
                    \ifnum\i<\j
                        \draw (res\i) -- (res\j);
                    \fi
                }
            
            \draw (res6) -- (res5);
            \draw (res6) -- (res1);
            \draw (res6) -- (res10);
            \draw (res10) -- (res5);
            \draw (res9) -- (res10);
            \draw (res9) -- (res2);
            \draw (res7) -- (res1);
            \draw (res7) -- (res3);
            \draw (res7) -- (res4);
            \draw (res8) -- (res4);
            \draw (res8) -- (res2);
            
            \foreach \i in {1,...,10}
                \draw[gray,line width=0.1] (res\i) -- (output);
            
            \begin{scope}[on background layer]
            \draw[draw=black,fill=pale_yellow,rounded corners=8pt]  ($(res1)+(-0.1,0.3)$) -- ($(res7)+(+0.2,+0.2)$) -- ($(res4)+(0.4,0)$) -- ($(res8)+(0.2,-0.2)$) -- ($(res2)+(0,-0.2)$) -- ($(res9)+(-0.2,-0.2)$) -- ($(res10)+(-0.3,0)$) -- ($(res6)+(-0.3,0)$) -- cycle;
            \end{scope}
            \node at (5.2, 1.4) {$\mathbf{W}_{\text{rec}}$};
            \node at (5, -1.6) {$\mathbf{s}_i(t)$};
            
            \draw[fill=white,opacity=0.8] (1.2,-0.75) rectangle (2.0,0.75) node[midway] {$\mathbf{W}_{\text{in}}$};
            \draw[fill=white,opacity=0.8] (8, -0.75) rectangle (8.8,0.75) node[midway] {$\mathbf{C}_{\text{out}}$};
        \end{tikzpicture}
        \caption{A diagram of an Echo State Network.}
        \label{fig:ESN}
    \end{figure}

    






