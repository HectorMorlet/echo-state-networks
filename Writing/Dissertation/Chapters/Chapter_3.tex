% !TEX root = ../HonoursThesisTemplate.tex

\chapter{Echo State Network with Ordinal Partition based readout switching}


Switch the readout ($C_{out}$) vector based on the ordinal partition.

When fitting the readout vectors, for each partition $p$:
\[
    (\mathbf{C_{out}})_p = (\mathbf{S}_p^T \mathbf{S}_p + \beta \mathbf{I}) \ \mathbf{S}_p^T \mathbf{Y}_p
\]
Where
\begin{itemize}
    \item $\mathbf{S}_p$ is $\mathbf{S}$ filtered to the states that results from data points with partition $p$. 
    \item $\mathbf{Y}_{p}$ is $\mathbf{Y}$ filtered to data points that follow immediately from data points with partition $p$.
\end{itemize}
And when infering the prediction at partition $p$:
\[
    \mathbf{y}(t) = (\mathbf{C}_{out})_{P(t)}\mathbf{s}(t)
\]
Where $P(t)$ is the partition of $x(t)$.

\begin{figure}
        
\centering
\begin{tikzpicture}[scale=1.2]
    \node[fill=col1,circle,inner sep=4pt,label=left:$\mathbf{x}(t)$] (input) at (0,0) {};
    \node[fill=black,circle,inner sep=2pt,label=right:$\mathbf{y}(t)$] (output) at (10,0) {};
    
    
    \coordinate (res_anchor) at (2,3);
    \node[fill=black,circle,inner sep=2pt] (res1) at ({5-0.5},{0.8}) {};
    \node[fill=black,circle,inner sep=2pt] (res2) at ({5},{-1}) {};
    \node[fill=black,circle,inner sep=2pt] (res3) at ({5},{0.15}) {};
    \node[fill=black,circle,inner sep=2pt] (res4) at ({5+1},{-0.1}) {};
    \node[fill=black,circle,inner sep=2pt] (res5) at ({5-0.75},{-0.5}) {};
    \node[fill=black,circle,inner sep=2pt] (res6) at ({5-1.25},{0.4}) {};
    \node[fill=black,circle,inner sep=2pt] (res7) at ({5+0.65},{0.75}) {};
    \node[fill=black,circle,inner sep=2pt] (res8) at ({5+1},{-0.9}) {};
    \node[fill=black,circle,inner sep=2pt] (res9) at ({5-1.25},{-1}) {};
    \node[fill=black,circle,inner sep=2pt] (res10) at ({5-1.75},{-0.25}) {};

    \foreach \i in {1,...,10}
        \draw[gray,line width=0.1] (input) -- (res\i);
    
    \foreach \i in {1,...,5}
        \foreach \j in {1,...,5} {
            \ifnum\i<\j
                \draw (res\i) -- (res\j);
            \fi
        }
    
    \draw (res6) -- (res5);
    \draw (res6) -- (res1);
    \draw (res6) -- (res10);
    \draw (res10) -- (res5);
    \draw (res9) -- (res10);
    \draw (res9) -- (res2);
    \draw (res7) -- (res1);
    \draw (res7) -- (res3);
    \draw (res7) -- (res4);
    \draw (res8) -- (res4);
    \draw (res8) -- (res2);
    
    \foreach \i in {1,...,10}
        \draw[gray,line width=0.1] (res\i) -- (output);
    
    \begin{scope}[on background layer]
    \draw[draw=black,fill=pale_yellow,rounded corners=8pt]  ($(res1)+(-0.1,0.3)$) -- ($(res7)+(+0.2,+0.2)$) -- ($(res4)+(0.4,0)$) -- ($(res8)+(0.2,-0.2)$) -- ($(res2)+(0,-0.2)$) -- ($(res9)+(-0.2,-0.2)$) -- ($(res10)+(-0.3,0)$) -- ($(res6)+(-0.3,0)$) -- cycle;
    \end{scope}
    \node at (5.2, 1.4) {$\mathbf{W}_{\text{rec}}$};
    \node at (5, -1.6) {$\mathbf{s}_i(t)$};
    
    \draw[fill=white,opacity=0.8] (1.2,-0.75) rectangle (2.0,0.75) node[midway] {$\mathbf{W}_{\text{in}}$};
    \draw[fill=col1,opacity=0.8] (8, -0.75) rectangle (8.8,0.75) node[midway, text=white] {$\mathbf{C}_{\text{out}}$};
\end{tikzpicture}

    \begin{tikzpicture}[scale=1.2]
        \node[fill=col2,circle,inner sep=4pt,label=left:$\mathbf{x}(t)$] (input) at (0,0) {};
        \node[fill=black,circle,inner sep=2pt,label=right:$\mathbf{y}(t)$] (output) at (10,0) {};
        
        
        \coordinate (res_anchor) at (2,3);
        \node[fill=black,circle,inner sep=2pt] (res1) at ({5-0.5},{0.8}) {};
        \node[fill=black,circle,inner sep=2pt] (res2) at ({5},{-1}) {};
        \node[fill=black,circle,inner sep=2pt] (res3) at ({5},{0.15}) {};
        \node[fill=black,circle,inner sep=2pt] (res4) at ({5+1},{-0.1}) {};
        \node[fill=black,circle,inner sep=2pt] (res5) at ({5-0.75},{-0.5}) {};
        \node[fill=black,circle,inner sep=2pt] (res6) at ({5-1.25},{0.4}) {};
        \node[fill=black,circle,inner sep=2pt] (res7) at ({5+0.65},{0.75}) {};
        \node[fill=black,circle,inner sep=2pt] (res8) at ({5+1},{-0.9}) {};
        \node[fill=black,circle,inner sep=2pt] (res9) at ({5-1.25},{-1}) {};
        \node[fill=black,circle,inner sep=2pt] (res10) at ({5-1.75},{-0.25}) {};
    
        \foreach \i in {1,...,10}
            \draw[gray,line width=0.1] (input) -- (res\i);
        
        \foreach \i in {1,...,5}
            \foreach \j in {1,...,5} {
                \ifnum\i<\j
                    \draw (res\i) -- (res\j);
                \fi
            }
        
        \draw (res6) -- (res5);
        \draw (res6) -- (res1);
        \draw (res6) -- (res10);
        \draw (res10) -- (res5);
        \draw (res9) -- (res10);
        \draw (res9) -- (res2);
        \draw (res7) -- (res1);
        \draw (res7) -- (res3);
        \draw (res7) -- (res4);
        \draw (res8) -- (res4);
        \draw (res8) -- (res2);
        
        \foreach \i in {1,...,10}
            \draw[gray,line width=0.1] (res\i) -- (output);
        
        \begin{scope}[on background layer]
        \draw[draw=black,fill=pale_yellow,rounded corners=8pt]  ($(res1)+(-0.1,0.3)$) -- ($(res7)+(+0.2,+0.2)$) -- ($(res4)+(0.4,0)$) -- ($(res8)+(0.2,-0.2)$) -- ($(res2)+(0,-0.2)$) -- ($(res9)+(-0.2,-0.2)$) -- ($(res10)+(-0.3,0)$) -- ($(res6)+(-0.3,0)$) -- cycle;
        \end{scope}
        \node at (5.2, 1.4) {$\mathbf{W}_{\text{rec}}$};
        \node at (5, -1.6) {$\mathbf{s}_i(t)$};
        
        \draw[fill=white,opacity=0.8] (1.2,-0.75) rectangle (2.0,0.75) node[midway] {$\mathbf{W}_{\text{in}}$};
        \draw[fill=col2,opacity=0.8] (8, -0.75) rectangle (8.8,0.75) node[midway, text=white] {$\mathbf{C}_{\text{out}}$};
    \end{tikzpicture}
    \caption{Echo State Network with readout switching.}
    \label{fig:ESN}
\end{figure}