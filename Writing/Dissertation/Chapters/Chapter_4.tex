% !TEX root = ../HonoursThesisTemplate.tex

\chapter{Echo State Network with Ordinal Partition based Sub-Reservoirs}

\begin{itemize}
    \item Restructure the reservoir into `subreservoirs' for each partition.
    \item Feed the input based on the ordinal partition.
    \item Weight the connections between subreservoirs according to the ordinal transition probabilities.
\end{itemize}

\begin{figure}
        
    \centering
    \begin{tikzpicture}%[scale=0.8]
        \node[fill=col1,circle,inner sep=4pt,label=left:$\mathbf{x}(t)$] (input) at (0,0) {};
        \node[fill=black,circle,inner sep=2pt,label=right:$\mathbf{y}(t)$] (output) at (10,0) {};
        


        % Layer 1
        
        % The nodes of the reservoir
        \coordinate (res_anchor) at (5,3);
        \node[fill=black,circle,inner sep=2pt] (res1) at ($(res_anchor) + (-0.5,0.8)$) {};
        \node[fill=black,circle,inner sep=2pt] (res2) at ($(res_anchor) + (0,-1)$) {};
        \node[fill=black,circle,inner sep=2pt] (res3) at ($(res_anchor) + (0,0.15)$) {};
        \node[fill=black,circle,inner sep=2pt] (res4) at ($(res_anchor) + (1,-0.1)$) {};
        \node[fill=black,circle,inner sep=2pt] (res5) at ($(res_anchor) + (-0.75,-0.5)$) {};
        \node[fill=black,circle,inner sep=2pt] (res6) at ($(res_anchor) + (-1.25,0.4)$) {};
        \node[fill=black,circle,inner sep=2pt] (res7) at ($(res_anchor) + (0.65,0.75)$) {};
        \node[fill=black,circle,inner sep=2pt] (res8) at ($(res_anchor) + (1,-0.9)$) {};
        \node[fill=black,circle,inner sep=2pt] (res9) at ($(res_anchor) + (-1.25,-1)$) {};
        \node[fill=black,circle,inner sep=2pt] (res10) at ($(res_anchor) + (-1.75,-0.25)$) {};
        
        % The internal connections in the reservoir
        \foreach \i in {1,...,5}
            \foreach \j in {1,...,5} {
                \ifnum\i<\j
                    \draw (res\i) -- (res\j);
                \fi
            }
        
        % More internal connections in the reservoir
        \draw (res6) -- (res5);
        \draw (res6) -- (res1);
        \draw (res6) -- (res10);
        \draw (res10) -- (res5);
        \draw (res9) -- (res10);
        \draw (res9) -- (res2);
        \draw (res7) -- (res1);
        \draw (res7) -- (res3);
        \draw (res7) -- (res4);
        \draw (res8) -- (res4);
        \draw (res8) -- (res2);
        
        % Reservoir background
        \begin{scope}[on background layer]
        \draw[draw=black,fill=col1_light,rounded corners=8pt]  ($(res1)+(-0.1,0.3)$) -- ($(res7)+(+0.2,+0.2)$) -- ($(res4)+(0.4,0)$) -- ($(res8)+(0.2,-0.2)$) -- ($(res2)+(0,-0.2)$) -- ($(res9)+(-0.2,-0.2)$) -- ($(res10)+(-0.3,0)$) -- ($(res6)+(-0.3,0)$) -- cycle;
        \end{scope}

        % Layer 2

        % The nodes of the reservoir
        \coordinate (res_anchor) at (5,0);
        \node[fill=black,circle,inner sep=2pt] (res11) at ($(res_anchor) + (-0.5,0.8)$) {};
        \node[fill=black,circle,inner sep=2pt] (res12) at ($(res_anchor) + (0,-1)$) {};
        \node[fill=black,circle,inner sep=2pt] (res13) at ($(res_anchor) + (0,0.15)$) {};
        \node[fill=black,circle,inner sep=2pt] (res14) at ($(res_anchor) + (1,-0.1)$) {};
        \node[fill=black,circle,inner sep=2pt] (res15) at ($(res_anchor) + (-0.75,-0.5)$) {};
        \node[fill=black,circle,inner sep=2pt] (res16) at ($(res_anchor) + (-1.25,0.4)$) {};
        \node[fill=black,circle,inner sep=2pt] (res17) at ($(res_anchor) + (0.65,0.75)$) {};
        \node[fill=black,circle,inner sep=2pt] (res18) at ($(res_anchor) + (1,-0.9)$) {};
        \node[fill=black,circle,inner sep=2pt] (res19) at ($(res_anchor) + (-1.25,-1)$) {};
        \node[fill=black,circle,inner sep=2pt] (res20) at ($(res_anchor) + (-1.75,-0.25)$) {};
        
        % The internal connections in the reservoir
        \foreach \i in {16,...,20}
            \foreach \j in {16,...,20} {
                \ifnum\i<\j
                    \draw (res\i) -- (res\j);
                \fi
            }
        
        % More internal connections in the reservoir
        \draw (res16) -- (res15);
        \draw (res16) -- (res11);
        % \draw (res16) -- (res20);
        % \draw (res20) -- (res15);
        % \draw (res19) -- (res20);
        % \draw (res19) -- (res12);
        % \draw (res17) -- (res11);
        % \draw (res17) -- (res13);
        % \draw (res17) -- (res14);
        % \draw (res18) -- (res14);
        % \draw (res18) -- (res12);
        
        % Reservoir background
        \begin{scope}[on background layer]
        \draw[draw=black,fill=col2_light,rounded corners=8pt]  ($(res11)+(-0.1,0.3)$) -- ($(res17)+(+0.2,+0.2)$) -- ($(res14)+(0.4,0)$) -- ($(res18)+(0.2,-0.2)$) -- ($(res12)+(0,-0.2)$) -- ($(res19)+(-0.2,-0.2)$) -- ($(res20)+(-0.3,0)$) -- ($(res16)+(-0.3,0)$) -- cycle;
        \end{scope}

        % Layer 3

        % The nodes of the reservoir
        \coordinate (res_anchor) at (5,-3);
        \node[fill=black,circle,inner sep=2pt] (res21) at ($(res_anchor) + (-0.5,0.8)$) {};
        \node[fill=black,circle,inner sep=2pt] (res22) at ($(res_anchor) + (0,-1)$) {};
        \node[fill=black,circle,inner sep=2pt] (res23) at ($(res_anchor) + (0,0.15)$) {};
        \node[fill=black,circle,inner sep=2pt] (res24) at ($(res_anchor) + (1,-0.1)$) {};
        \node[fill=black,circle,inner sep=2pt] (res25) at ($(res_anchor) + (-0.75,-0.5)$) {};
        \node[fill=black,circle,inner sep=2pt] (res26) at ($(res_anchor) + (-1.25,0.4)$) {};
        \node[fill=black,circle,inner sep=2pt] (res27) at ($(res_anchor) + (0.65,0.75)$) {};
        \node[fill=black,circle,inner sep=2pt] (res28) at ($(res_anchor) + (1,-0.9)$) {};
        \node[fill=black,circle,inner sep=2pt] (res29) at ($(res_anchor) + (-1.25,-1)$) {};
        \node[fill=black,circle,inner sep=2pt] (res30) at ($(res_anchor) + (-1.75,-0.25)$) {};
        
        % The internal connections in the reservoir
        \foreach \i in {24,...,28}
            \foreach \j in {24,...,28} {
                \ifnum\i<\j
                    \draw (res\i) -- (res\j);
                \fi
            }
        
        % More internal connections in the reservoir
        % \draw (res26) -- (res25);
        % \draw (res26) -- (res21);
        \draw (res26) -- (res30);
        \draw (res30) -- (res25);
        \draw (res29) -- (res30);
        \draw (res29) -- (res22);
        % \draw (res27) -- (res21);
        % \draw (res27) -- (res23);
        % \draw (res27) -- (res24);
        % \draw (res28) -- (res24);
        % \draw (res28) -- (res22);
        
        % Reservoir background
        \begin{scope}[on background layer]
        \draw[draw=black,fill=col3_light,rounded corners=8pt]  ($(res21)+(-0.1,0.3)$) -- ($(res27)+(+0.2,+0.2)$) -- ($(res24)+(0.4,0)$) -- ($(res28)+(0.2,-0.2)$) -- ($(res22)+(0,-0.2)$) -- ($(res29)+(-0.2,-0.2)$) -- ($(res30)+(-0.3,0)$) -- ($(res26)+(-0.3,0)$) -- cycle;
        \end{scope}

        % Dotted connections between layers
        \foreach \i in {1,...,10} {
            \pgfmathtruncatemacro{\j}{\i+10}
            \pgfmathtruncatemacro{\k}{\i+20}
            \draw[dotted] (res\i) -- (res\j);
            \draw[dotted] (res\i) -- (res\k);
            \draw[dotted] (res\j) -- (res\k);
        }


        % The input connections
        \foreach \i in {1,...,10}
            \draw[gray,line width=0.1, path fading=fadeRL] (input) -- (res\i);
        
        % Output connections
        \foreach \i in {1,...,30} {
            \draw[gray, line width=0.1, path fading=fadeLR] (res\i) -- (output);
        }

        \draw[fill=white,opacity=0.8] (1.2,0.25) rectangle (2.0,1.75) node[midway] {$\mathbf{W}_{\text{in}}$};
        \draw[fill=white,opacity=0.8] (8, -0.75) rectangle (8.8,0.75) node[midway] {$\mathbf{C}_{\text{out}}$};
    \end{tikzpicture}
    \caption{Ordinally Partitioned Echo State Network.}
    \label{fig:ESN}
\end{figure}

For example, consider a partitioned Echo State Network with $3$ partitions (not actually possible).
    
With 3 partitions, the time series has the following transition probabilities:

\begin{table}[]
    \centering
    \begin{tabular}{c|cccccc}
        & \tikz\draw[fill=col1,draw=col1] (0,0) circle (0.9ex); 
        & \tikz\draw[fill=col2,draw=col2] (0,0) circle (0.9ex); 
        & \tikz\draw[fill=col3,draw=col3] (0,0) circle (0.9ex); \\ \hline
        \tikz\draw[fill=col1,draw=col1] (0,0) circle (0.9ex); & 0.7 & 0.1 & 0.2 \\
        \tikz\draw[fill=col2,draw=col2] (0,0) circle (0.9ex); & 0.9 & 0.1 & 0 \\
        \tikz\draw[fill=col3,draw=col3] (0,0) circle (0.9ex); & 0.1 & 0 & 0.8
    \end{tabular}
    \label{tab:transition_probabilities}
    \caption{Example transition probabilities}
\end{table}
    









    Let $k_{part}$ be the number of nodes in each partition's sub reservoir,
    and let $\mathbf{W}_{p,q}$ refer to the submatrix of $\mathbf{W}_{rec}$ with indices:
    
    \begin{align*}
        \text{Rows:}\quad& \{ pk_{part} + 1, pk_{part} + 2, \dots, (p+1)k_{part} \} \\
        \text{Columns:}\quad& \{ qk_{part} + 1, qk_{part} + 2, \dots, (q+1)k_{part} \}
    \end{align*}
    

    % Then, we define \(\mathbf{W}_{rec}\) in block form as \(\mathbf{W}_{p,q}\), where each block \(\mathbf{W}_{p,q}\) is the submatrix of \(\mathbf{W}_{rec}\) whose rows are in \(\Omega_p\) and columns are in \(\Omega_q\). The weight matrix is then given by:
    Then $\mathbf{W}_{rec}$ is given by:

    \[
    \mathbf{W}_{p,q} =
    \begin{cases}
    \mathbf{I} P(p,q), & p \neq q, \\
    \mathbf{W}_{ER}, & p = q,
    \end{cases}
    \]

    where 
    \begin{itemize}
        \item \(\mathbf{I}\) is the \( k_{part} \times k_{part} \) identity matrix
        \item \( P(p,q) \) is the probability of transitioning from partition \( p \) to partition \( q \)
        \item $\mathbf{W}_{ER}$ is an Erdos-Renyi randomly instantiated network.
    \end{itemize}
    








    Example:

    For a $k_{layer}$ of $4$ and $3$ partitions (not actually possible).

    \begin{center}
    $W_{rec} = $
    \[
    \scalebox{0.75}{$
    \begin{bmatrix}
    0.12 & 0 & 0.34 & -0.67 & 0.78 & 0 & -0.23 & 0.11 & -0.56 & 0 & 0.47 & 0 \\
    -0.44 & 0.23 & 0 & 0 & -0.76 & 0.54 & 0 & -0.34 & 0 & 0.56 & 0.89 & 0 \\
    0.23 & 0.78 & 0 & 0 & 0.34 & 0.89 & 0 & 0.45 & 0.67 & 0 & -0.89 & -0.75 \\
    0 & 0.34 & 0.89 & 0 & 0.23 & 0 & -0.67 & 0.12 & 0 & 0.56 & 0 & -0.78 \\
    0.67 & 0.12 & 0 & 0.45 & 0 & 0.78 & 0 & 0.89 & 0.12 & 0 & 0.23 & 0 \\
    0 & 0.56 & 0.89 & 0 & 0.23 & 0 & 0.67 & 0 & 0.34 & 0.78 & 0 & 0 \\
    0.45 & 0 & 0.89 & 0.23 & 0 & 0.56 & 0.78 & 0 & 0.12 & 0 & 0.45 & 0.1 \\
    0.78 & 0.12 & 0 & 0 & 0.34 & 0 & 0.45 & 0.56 & 0 & 0.89 & 0 & 0 \\
    0.89 & 0.23 & 0 & 0.56 & 0 & 0.89 & 0 & 0 & 0.34 & 0 & 0.78 & 0.24 \\
    0.34 & 0 & 0.78 & 0 & 0.12 & 0 & 0 & 0.67 & 0 & 0.12 & 0 & 0 \\
    0.78 & 0.12 & 0 & 0.34 & 0 & 0.78 & 0.89 & 0 & 0.23 & 0.56 & 0 & -0.95 \\
    0 & 0.89 & 0.23 & 0 & 0.56 & 0 & 0.89 & 0.12 & 0 & 0.34 & 0 & 0
    \end{bmatrix}
    $}
    \]
    \end{center}
    











    Example:
    
    For a $k_{layer}$ of $4$ and $3$ partitions (not actually possible).

    \begin{center}
    $W_{rec} = $
    \[
    \scalebox{0.75}{$
    \begin{bmatrix}
    0.12 & 0 & 0.34 & -0.67 & 0 & 0 & 0 & 0 & 0 & 0 & 0 & 0 \\
    -0.44 & 0.23 & 0 & 0 & 0 & 0 & 0 & 0 & 0 & 0 & 0 & 0 \\
    0.23 & 0.78 & 0 & 0 & 0 & 0 & 0 & 0 & 0 & 0 & 0 & 0 \\
    0 & 0.34 & 0.89 & 0 & 0 & 0 & 0 & 0 & 0 & 0 & 0 & 0 \\
    0 & 0 & 0 & 0 & 0 & 0.78 & 0 & 0.89 & 0 & 0 & 0 & 0 \\
    0 & 0 & 0 & 0 & 0.23 & 0 & 0.67 & 0 & 0 & 0 & 0 & 0 \\
    0 & 0 & 0 & 0 & 0 & 0.56 & 0.78 & 0 & 0 & 0 & 0 & 0 \\
    0 & 0 & 0 & 0 & 0.34 & 0 & 0.45 & 0.56 & 0 & 0 & 0 & 0 \\
    0 & 0 & 0 & 0 & 0 & 0 & 0 & 0 & 0.34 & 0 & 0.78 & 0.24 \\
    0 & 0 & 0 & 0 & 0 & 0 & 0 & 0 & 0 & 0.12 & 0 & 0 \\
    0 & 0 & 0 & 0 & 0 & 0 & 0 & 0 & 0.23 & 0.56 & 0 & -0.95 \\
    0 & 0 & 0 & 0 & 0 & 0 & 0 & 0 & 0 & 0.34 & 0 & 0
    \end{bmatrix}
    $}
    \]
    \end{center}
    









    Example:
    
    For a $k_{layer}$ of $4$ and $3$ partitions (not actually possible).

    \begin{center}
    $W_{rec} = $
    \[
    \scalebox{0.75}{$
    \begin{bmatrix}
        0.12 & 0 & 0.34 & -0.67 & 0.1 & 0 & 0 & 0 & 0.2 & 0 & 0 & 0 \\
        -0.44 & 0.23 & 0 & 0 & 0 & 0.1 & 0 & 0 & 0 & 0.2 & 0 & 0 \\
        0.23 & 0.78 & 0 & 0 & 0 & 0 & 0.1 & 0 & 0 & 0 & 0.2 & 0 \\
        0 & 0.34 & 0.89 & 0 & 0 & 0 & 0 & 0.1 & 0 & 0 & 0 & 0.2 \\
        0.9 & 0 & 0 & 0 & 0 & 0.78 & 0 & 0.89 & 0 & 0 & 0 & 0 \\
        0 & 0.9 & 0 & 0 & 0.23 & 0 & 0.67 & 0 & 0 & 0 & 0 & 0 \\
        0 & 0 & 0.9 & 0 & 0 & 0.56 & 0.78 & 0 & 0 & 0 & 0 & 0 \\
        0 & 0 & 0 & 0.9 & 0.34 & 0 & 0.45 & 0.56 & 0 & 0 & 0 & 0 \\
        0.1 & 0 & 0 & 0 & 0 & 0 & 0 & 0 & 0.34 & 0 & 0.78 & 0.24 \\
        0 & 0.1 & 0 & 0 & 0 & 0 & 0 & 0 & 0 & 0.12 & 0 & 0 \\
        0 & 0 & 0.1 & 0 & 0 & 0 & 0 & 0 & 0.23 & 0.56 & 0 & -0.95 \\
        0 & 0 & 0 & 0.1 & 0 & 0 & 0 & 0 & 0 & 0.34 & 0 & 0
    \end{bmatrix}
    $}
    \]
    \end{center}













    To only feed the input into the relevant layer, we mask all values of $\textbf{W}_{in}$ except those in the relevant partition.

    Given a randomly instantiated input vector $\textbf{W}_{all}$ of size $k_{part}O$ where $O$ is the number of partitions, we can define $\textbf{W}_{in}$ as a function of $p_t$, the partition of the input at time $t$:

    \[
        \textbf{W}_{in}(p_t)_i = \begin{cases}
            (\textbf{W}_{all})_i, & k_{part}(p_t-1) < i \leq k_{part}p_t\\
            0,                   & otherwise.
        \end{cases}
    \]

    % \begin{center}
    Example:
        \[
        \scalebox{0.75}{$
        \textbf{W}_{all} = \begin{bmatrix}
            0.12 & 0.5 & 0.34 & -0.67 & 0.1 & -0.43 & 0.98 & -0.64 & 0.2 & 0.45 & 0.2 & -0.45
        \end{bmatrix}
        $}
        \]
        \[
        \scalebox{0.75}{$
        \textbf{W}_{in}(1) = \begin{bmatrix}
            0.12 & 0.5 & 0.34 & -0.67 & 0 & 0 & 0 & 0 & 0 & 0 & 0 & 0
        \end{bmatrix}
        $}
        \]
        \[
        \scalebox{0.75}{$
        \textbf{W}_{in}(2) = \begin{bmatrix}
            0 & 0 & 0 & 0 & 0.1 & -0.43 & 0.98 & -0.64 & 0 & 0 & 0 & 0
        \end{bmatrix}
        $}
        \]
        \[
        \scalebox{0.75}{$
        \textbf{W}_{in}(3) = \begin{bmatrix}
            0 & 0 & 0 & 0 & 0 & 0 & 0 & 0 & 0.2 & 0.45 & 0.2 & -0.45
        \end{bmatrix}
        $}
        \]
    % \end{center}
